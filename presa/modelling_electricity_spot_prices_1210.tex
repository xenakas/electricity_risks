%!TEX TS-program = xelatex
% Исходная версия шаблона --- 
% https://www.writelatex.com/coursera/latex/5.1
\documentclass[c, dvipsnames]{beamer}  % [t], [c], или [b] --- вертикальное 
%\documentclass[handout, dvipsnames, c]{beamer} % Раздаточный материал (на слайдах всё сразу)
\input{pre}
\title[   Моделирование цен на электричество ]{Моделирование спотовых цен на электроэнергию    на оптовом рынке  в России }
%\subtitle{Защита выпускной квалификационной работы}


%Stochastic Factors Affecting Commodity Prices: the Case of Russia

 \usepackage{amsmath}

\author[Касьянова Ксения]{Касьянова Ксения \\ \smallskip \scriptsize  }

%\superviser{ Научный руководитель }

%\author[Имя автора]{Имя автора \\ \smallskip \scriptsize \href{mailto:author@ranepa.ru}{author@ranepa.ru} \\ \smallskip  \href{http://ranepa.ru}{http://ranepa.ru} }

\institute[ИОРИ РАНХиГС]{ \uppercase{
  Институт отраслевых рынков и инфраструктуры}}
\date{}



\titlegraphic{\includegraphics[scale=0.5]{logo1}}
\titlegraphicii{\includegraphics[scale=0.5]{logo2}}

\begin{document}

\frame[plain]{\titlepage}	% Титульный слайд

\section{Введение}

\begin{frame}[shrink=3]
\frametitle{Цели и задачи} 



\begin{block}{Цель:}
	\begin{itemize}
		
		\item разработка модели ценообразования на оптовом рынке электроэнергии (РСВ), учитывающей особенности российского рынка; 
		
		\item оценка влияния принятия различных решений  и изменений факторов на цену на электричество и финансовые риски участников рынка электрической энергии. 
		
	\end{itemize}

	
\end{block}

\begin{block}{Гипотеза:}
	\begin{itemize}
		
		\item построив математическую  модель, описывающую  цену на электроэнергию как диффузионно-скачкообразный процесс,  учитывающую также экономические (фундаментальные) факторы, влияющие на спрос и предложение на рынке электроэнергии, можно  проследить, как отразится их изменение   на финансовые риски этого актива, а также объяснить различия динамики цен в ценовых зонах.
		
%		\item стандартные модели оценки риска не подходят для российской рынка, риск недооценен, добавление дополнительных факторов позволит получить более точные оценки риска. 
	
		%		\item   фьючерсы недавно торгуются, в России биржа электричества молодая, Россия необычная страна)))).  
		
		
	\end{itemize}
\end{block}




%альтернативы электричеству:  про нефть не особо, все внутренние товарные биржи ненастоящие и нужны ФАС для оценки, если внешние - не понятно какие котировки брать, если общие, то не понятно, какое влияние оказывает на наш рынок, если русские - не понятно кто торгует, в чем смысл, кто виноват и что делать


\end{frame}

\begin{frame}[shrink=3]
\frametitle{Цели и задачи} 


%модель строится с целью иметь возможность проследить, что произойдет на рынке при влиянии на какие-то из встроенных в модель факторов 

% надо понимать, что без учета стохастической компоненты это невозможно, так как цена на эл-во нестанционарный процесс и при оценке  какого-либо  влияния факторов на цены велика вероятность, что колебания в этот момент были случайными  

% боремся с этим, во-первых, добавляя в модель стохастическую компоненту, во-вторых, используя байесовский подход при оценке этой модели мы сможем получить результат в виде "с какой вероятностью такие изменения на рынке были вызваны именно внешний воздейтсвием на фактор, а не случайными изменениями цены"



\begin{block}{Задачи:}


	\begin{itemize}
		\item  описание принципа функционирования оптового рынка электроэнергии;
		
		\item  выявление факторов  влияющих на цены на электричество, особенностей российского рынка; 
		
		
		
%		\item  сравнение классических моделей оценки риска (VaR) на российских данных (можно попробовать применить к американским/европейским);		

%		цель посчитать VaR в класическом  варианте, сравнить с реальными рисками, сказать, что риск был недоценен, улучшить с помощью изменения распределения
				  
		\item  выбор подходящей модели, способной учесть неодинаковое    влияние факторов на различные компоненты процесса (тренд, сезонность и стохастические компоненты);
		  
 
		\item  оценивание моделей, сравнение с бенчмарк-моделями (не байесовскими/не стохастическими);
		
		\item  выбор событий/решений/политик повлиявших на факторы, включенные в модель, сравнение рисков до/после. 
%		  цены и оценка риска на российском рынке электричества с учетом этих факторов, сравнение с классической моделью,  выявление причин, по которым  в какие-то моменты риски были завышены/занижены.

% 		\item  проверка коинтегрированности факторов с ценами;

%		\item  добавление факторов в модели оценки риска (например, вместо нормального распределения в VaR моделях использовать скорректированное распределение, учитывающее эти факторы);
%\item  сравнение модели с базовыми: определить, удалось ли решить проблему с неправильными оценками риска. 		
		
	\end{itemize}
	
\end{block}

\begin{block}{Актуальность:}
	\begin{itemize}
		
		\item около 72\% производимой электроэнергии продается на рынке на сутки вперед (РСВ);
%		\item необходимость верно оценивать финансовые риски возникает у производителей электричества; 
		% (1 час - 1 день),    финансовых посредников  торгующих контрактами на электроэнергию (1 день - 1 месяц), инвесторов (1 месяц - 1 год).
		\item прямая связь с задачей ценообразования производных финансовых инструментов, необходимых для хеджирования  финансовых рисков. 
		%		\item   фьючерсы недавно торгуются, в России биржа электричества молодая, Россия необычная страна)))).  
		
		
	\end{itemize}
\end{block}


\end{frame}


\begin{frame}[shrink=5]



\frametitle{Содержание} 


\begin{enumerate}
	\item принцип работы ОРЭМ
	\item выбор факторов, влияющих на цены:
	\begin{enumerate}
		\item особенности рынка электричества
		\item особенности российского рынка
		\item выбор ценовых факторов, включаемых в модель
	\end{enumerate}
	\item моделирование цены как диффузионно-скачкообразного процесса  
\end{enumerate}



\end{frame}



\section{Российский рынок}




\begin{frame}[shrink=5]



\frametitle{Российский рынок электроэнергии и мощности} 



\begin{figure}
	\centering
	\includegraphics[width=0.8\linewidth]{screenshot004}
	%	\caption{}
	\label{fig:screenshot004}
\end{figure}


\end{frame}



\begin{frame}[shrink=5]
\frametitle{Российский рынок электроэнергии и мощности} 
\framesubtitle{Специфика ценообразования на российском рынке} 

Тариф для конечного потребителя на электроэнергию и мощность формируется на основе пяти составляющих:
\begin{itemize}
	\item 
	цена электроэнергии (цена покупки электроэнергии на оптовом рынке или у розничного генератора);
	\item  цена мощности (цена покупки мощности энергосбытовой компанией на оптовом рынке или у розничного генератора);
	\item  цена передачи по сети с дифференциацией по уровню напряжения: тарифы ФСК на передачу по магистральным сетям, тарифы МРСК на передачу по сетям среднего напряжения и тариф ТСО на передачу по сетям низкого напряжения;
	\item  инфраструктурные платежи: плата за услуги СО ЕЭС, АТС, ЦФР. Размер платы регулируется ФАС Россиии Ассоциацией «НП Совет рынка»;
	\item   надбавка сбытовых компаний  (кроме участников оптового рынка).
	
\end{itemize}
\end{frame}



\begin{frame}[shrink=5]
\frametitle{Российский оптовый рынок электричества} 

%\begin{figure}
%	\centering
%	\includegraphics[width=0.7\linewidth]{screenshot003}
%	\caption{ Ценовые зоны  }
%	\label{fig:screenshot003}
%\end{figure}

\begin{figure}
	\centering
	\includegraphics[width=1\linewidth]{screenshot014}
	\caption{ Ценовые зоны. Источник: АТС }
	\label{fig:screenshot013}
\end{figure}



\end{frame}







\begin{frame}[shrink=5]
\frametitle{Российский оптовый рынок электричества} 


\begin{figure}
	\centering
	\includegraphics[width=0.8\linewidth]{screenshot16}
	\caption{ Зоны свободного перетока мощности оптового рынка.  Источник: ПЕРЕТОК.РУ }
	\label{fig:screenshot015}
\end{figure}

%https://peretok.ru/articles/strategy/14495/

\footnotesize{* С точки зрения энергорынка, разделение на ЗСП по-прежнему учитывается только при определении вынужденной генерации, при этом используется базовый перечень ЗСП.}


\end{frame}




\begin{frame}[shrink=5]
\frametitle{Российский оптовый рынок электричества} 


\begin{itemize}
	\item При формировании поузловых модельных пар <цена-количество> СО  определяет на основе ценовых заявок на планирование объемов производства в отношении групп точек поставки (ГТП) генерации объемы электрической энергии, заявленные Участниками оптового рынка в отношении каждой ГТП генерации к продаже на сутки вперед, и из этих объемов выделяет объемы (часть объемов) электрической энергии, содержащиеся в зарегистрированных двусторонних договорах, и которые подлежат включению в торговый график в приоритетном порядке, и формирует на эти объемы ценопринимающую часть вместо условий третьего и четвертого приоритета для модельных пар, сформированных АТС при проведении конкурентного отбора на сутки вперед согласно 3-му и 4-му буллитам п.1.2;
	\item если ГТП относится к нескольким узлам расчетной модели, СО распределяет объемы электрической энергии, содержащиеся в каждой паре «цена – количество» в ценовой заявке в соответствии с коэффициентами или формулами отнесения объемов к каждому узлу согласно Методике формирования входных и выходных данных при  проведении конкурентного отбора БР
\end{itemize}

Источник:
Регламент проведения конкурентного  отбора заявок для балансирования системы

\end{frame}



\begin{frame}[shrink=5]
\frametitle{Российский оптовый рынок электричества} 

СО проводит конкурентный отбор БР и определение диспетчерских объемов, индикаторов стоимости и цен балансирования в соответствии с Математической моделью расчета диспетчерских объемов электрической энергии, индикаторов и цен на балансирование вверх (вниз) в результате конкурентного отбора ценовых заявок БР так, чтобы:


\begin{itemize}
\item в диспетчерские объемы были включены все объемы электрической энергии, не превышающие установленных пределов, относящиеся к соответствующему узлу
\item индикатор в данном узле был не меньше цены, указанной Участником оптового рынка в ценовой заявке на планирование объема отрицательного потребления в отношении ГТП потребления с регулируемой нагрузкой по объекту управления или в ценовой заявке на планирование объема производства в отношении ГТП генерации за объем электрической энергии
\item индикаторы во всех узлах расчетной модели отличались на стоимость нагрузочных потерь электрической энергии и системных ограничений.

Источник:
Регламент проведения конкурентного  отбора заявок для балансирования системы


\end{itemize}
\end{frame}


\begin{frame}[shrink=5]
\frametitle{Российский оптовый рынок электричества} 
\framesubtitle{Специальные случаи расчета результатов конкурентного отбора} 

Если при проведении конкурентного отбора БР для определенного операционного часа в некоторой группе узлов расчетной модели:

\begin{itemize}
\item  приняты только ценопринимающие объемы в заявках на продажу индикаторы в этой группе узлов считаются равными нулю;
\item принята заявка с четвертой (дополнительной) ступенью, с ценой или с модельной ценой, равной десяти тарифам на электроэнергию, для определения индикаторов стоимости на данный час проводится дополнительный расчет, обеспечивающий вычисление индикаторов стоимости по ценам, не превышающих указанной модельной цены и максимальной из цен, указанных в принятых третьих ступенях ценовых заявок участников;
\item  оказывается, что для какого-либо часа объемов производства недостаточно для формирования ПБР, удовлетворяющего в этот час остальным СО вправе изменить состав выбранного оборудования;
\item 	 не удается выполнить действия предыдущего подпункта  СО имеет право в установленном порядке ввести ограничения потребления и/или изменить ограничения на перетоки по сечениям экспортно-импортных операций.
\end{itemize}

Источник:
Регламент проведения конкурентного  отбора заявок для балансирования системы



\end{frame}


\begin{frame}[shrink=5]
\frametitle{Российский оптовый рынок электричества} 



%\begin{figure}
%	\centering
%	\includegraphics[width=0.7\linewidth]{screenshot016}
%	\caption{Формирование ценовой заявки поставщика 
%		для конкурентного отбора  РСВ и БР. Источник: E.ON  }
%	\label{fig:screenshot016}
%\end{figure}

Формирование ценовой заявки поставщика 	для конкурентного отбора  РСВ и БР. 

\

\vfil
\hfil\hfil\includegraphics[height=3cm]{screenshot016}\hfil\hfil
\includegraphics[height=2.8cm]{screenshot017}\newline

\vfil
\hfil\hfil\includegraphics[height=2cm]{screenshot018}\hfil\hfil
\includegraphics[height=2cm]{screenshot019}
\hfil\hfil\includegraphics[height=2cm]{screenshot020}\newline

\

Источник: E.ON

\end{frame}



\begin{frame}[shrink=5]
\frametitle{Российский оптовый рынок электричества} 

\begin{figure}
	\centering
	\includegraphics[width=1\linewidth]{screenshot021}
	\caption{ Определение и фиксация объемов поставки и потребления. Источник: E.ON }
	\label{fig:screenshot015}
\end{figure}




\end{frame}



\begin{frame}[shrink=5]
\frametitle{Российский оптовый рынок электричества} 
\framesubtitle{Параметры спроса и предложения электрической энергии} 



\begin{table}[]
	\begin{tabular}{lllll}
		
		ЦЗ: &  & Европа &  &  \\
		Дата: &  & 12.07.2018 &  &  \\
		Чаc: &  & 10 &  &  \\
		&  &  &  &  \\
		№ & ЦЗ на покупку &  & ЦЗ на продажу &  \\
		& Цена (руб./МВтч) & Объем (МВтч) & Цена (руб./МВтч) & Объем (МВтч) \\
		1 & * & 88253.3619098945 & * & 83264.25 \\
		2 & 100 & 1 & 271 & 3 \\
		3 & 1900 & 33.705 & 300 & 54 \\
		4 & 2300 & 41 & 390 & 0.99999999999976 \\
		5 &  &  & 427 & 1 \\
		6 &  &  & 429 & 49.66 \\
		7 &  &  & 444 & 1 \\
		… &  &  & … & … \\
		208 &  &  & 3000 & 32 \\
		209 &  &  & 3500 & 0.5 \\
		210 &  &  & 3860 & 4.375 \\
		211 &  &  & 5490 & 27 \\
		212 &  &  & 5842 & 2 \\
		213 &  &  & 6100 & 9.5 \\
		214 &  &  & 13260 & 102.652
	\end{tabular}
\end{table}

Источник: АТС

\end{frame}


\begin{frame}[shrink=5]
\frametitle{Российский оптовый рынок электричества} 


Плановые почаcовые объемы потребления, МВтЧ.: 

\begin{table}[]
	\begin{tabular}{llllll}
		Дата & 17.07.2019 &  &  &  &  \\
		ЦЗ: & Европа &  &  &  &  \\
		&  &  &  &  &  \\
		Час &  &  &  &  &  \\
		& … & 11-12 & 12-13 & 13-14 & … \\
		ГП & … & 52507.568 & 52146.723 & 52444.545 & … \\
		не ГП & … & 31586.293 & 31625.01 & 31707.908 & …
	\end{tabular}
\end{table}

Источник: АТС


\

\footnotesize{* Гарантирующий поставщик обязан заключить с любым обратившимся к нему физическим или юридическим лицом, находящимся в зоне его деятельности, договор энергоснабжения (купли-продажи (поставки) электрической энергии (мощности)).
}


\end{frame}




\begin{frame}[shrink=5]
\frametitle{Российский оптовый рынок электричества} 

\begin{figure}
	\centering
	\includegraphics[width=0.8\linewidth]{screenshot022}
	\caption{ Математическая модель расчета узловых цен по методике АТС. Источник: Б.Г. Булатов, В.О. Каркунов (2009)  }
	\label{fig:screenshot015}
\end{figure}


%https://www.mbureau.ru/tag/uzlovye-ceny

\end{frame}



\begin{frame}[shrink=5]
\frametitle{Данные} 


Данные по ценам на электричество за каждый час, начиная с 1.08.2013 по двум ценовым зонам: 

\begin{itemize}
	\item Объем полного планового потребления, МВт.ч
	\item Индекс равновесных цен на покупку электроэнергии, руб./МВт.ч.
	\item Объем покупки по регулируемым договорам, МВт.ч
	\item  Объем покупки на РСВ, МВт.ч	
	\item Объем продажи в обеспечение РД, МВт.ч	
	%	\item  Максимальный индекс равновесной цены, руб./МВт.ч	\item Минимальный индекс равновесной цены, руб./МВт.ч
	%	
	
\end{itemize}

Источник: \href{https://www.atsenergo.ru/results/rsv}{АТС}




%SPIMEX:
%
%- URALS
%
%- Биржевые индексы цен нефтепродуктов: 92, 95, ДТ, СУГ.
%
%НТБ:
%
%- Зерновые
%
%http://www.ntb.moex.com/ru/archive/Zakupki2016

% Please add the following required packages to your document preamble:
% \usepackage{multirow}
%		
%\begin{table}[]
%	
%	\small\centering\setlength{\extrarowheight}{0.25em}
%	
%	\begin{tabular}{   >{\centering\footnotesize}p{8em} 
%			>{\centering\footnotesize}p{4em} 
%			>{\centering\footnotesize}p{4em} 
%			>{\centering\footnotesize}p{4em} 
%			>{\centering\footnotesize\arraybackslash}p{6em} }\hline
%
%		
%		
%\multirow{2}{*}{\begin{tabular}[c]{@{}c@{}}Агрегированный\\ ряд\end{tabular}} & \multicolumn{3}{c}{Ряды второго уровня} & \multirow{2}{*}{\begin{tabular}[c]{@{}c@{}}Число рядов \\ третьего уровня\end{tabular}} \\
%& по регионам & по типам & по кластерам &  \\\hline
%ВВП ЕС & 28 & 10 & 25 & 280 \\
%ВВП США & 50 & 21 & 25 & 1050 \\
%ЕП РФ & 80 & 4 & 25 & 320\\\hline
%
%	\end{tabular}
%\end{table}
%
%
%\begin{table}[]
%	
%	\small\centering\setlength{\extrarowheight}{0.25em}
%	
%	
%	
%	\begin{tabular}{   >{\centering\footnotesize}p{8em} 
%			>{\centering\footnotesize}p{5em} 
%			>{\centering\footnotesize}p{5em} 
%			>{\centering\footnotesize}p{4em} 
%			>{\centering\footnotesize\arraybackslash}p{4em} }\hline
%		
%		\multirow{2}{*}{\begin{tabular}[c]{@{}c@{}}Агрегированный\\ ряд\end{tabular}} & \multirow{2}{*}{Сезонность} & \multirow{2}{*}{\begin{tabular}[c]{@{}c@{}}Число\\ наблюдений\end{tabular}} & \multicolumn{2}{c}{Кросс-валидация} \\
%		&  &  & \begin{tabular}[c]{@{}c@{}}число\\ подвыборок\end{tabular} & \begin{tabular}[c]{@{}c@{}}ширина\\ окна\end{tabular} \\\hline
%		ВВП ЕС & 4 & 75 & 6 & 48 \\
%		ВВП США & 4 & 54 & 6 & 28 \\
%		ЕП РФ & 12 & 157 & 5 & 84\\\hline
%	\end{tabular}
%\end{table}
%


\end{frame}

\begin{frame}[shrink=5]
\frametitle{Российский оптовый рынок электричества} 

\begin{figure}
	\centering
	\includegraphics[width=0.7\linewidth]{screenshot009}
	\caption{Спотовые цены (усредненные за день) для 1 и 2 ценовой зон, руб./МВт.ч}
	\label{fig:screenshot009}
\end{figure}


\end{frame}




\begin{frame}[shrink=5]
\frametitle{Российский оптовый рынок электричества} 

\begin{equation}\label{key}
corr(p_1,p_2)  =  0.07
\end{equation}

\begin{figure}
	\centering
	\includegraphics[width=0.7\linewidth]{screenshot010}
	\caption{Спотовые цены (усредненные за месяц) для 1 и 2 ценовой зон, руб./МВт.ч.}
	\label{fig:screenshot009}
\end{figure}




\end{frame}




\begin{frame}[shrink=5]
\frametitle{Российский оптовый рынок электричества} 

\begin{figure}
	\centering
	\includegraphics[width=0.7\linewidth]{screenshot011}
	\caption{Разница дневных объемов планового предложения и потребления электроэнергии для 1 и 2 ценовой зон, МВт.ч.}
	\label{fig:screenshot011}
\end{figure}




\end{frame}



\section{Факторы}


\begin{frame}[shrink=5]

\frametitle{Факторы  влияющие на цены на электричество} 

\framesubtitle{Основные экономические модели ценообразования на рынке электричества} 


\begin{itemize}
	\item  моделирование с учетом фундаментальных факторов (физических/экономических)
	\item модели типа Курно (в результате - цены выше чем в действительности)
	\item моделирование совокупной функции предложения (необходимо решить систему дифференциальных уравнений, вычислительно затратно, не уделяется внимание резким всплескам) 
	\item  моделирование поведения групп агентов  (необходимо для выявления сложных зависимостей, применяется совместно с другими моделями, высокие риски моделирования, так как согласование с теоретической моделью  и эмпирическими наблюдениями сильно зависит от предпосылок и понимания настоящей структуры рынка)
	
\end{itemize}

\end{frame}



\begin{frame}[shrink=5]
\frametitle{Факторы  влияющие на цены на электричество} 
\framesubtitle{Особенности рынка электричества} 

\begin{itemize}
	\item невозможность хранения => проблема обязательства энергоустановки (unit commitment), учитывается при моделировании цены фьючерсного контракта (так как невозможно открыть короткую позицию). 
	\item проблема с ограничениями ЛЭП (проблема решается единым оператором), возможность перенапряжения сети (в таком случае локальные цены отличаются от общеустановленных по системе) 
	\item цены на электричество определяются на РСВ, т.е. отсутствует непрерывность торговли, решения на все сутки принимаются на основании одного и того же информационного множества
	\item невозможность перераспределить волатильность цен по производственной цепочке 
	\item  цены имеют три уровня циклических колебаний: ежедневная, недельная, годовая (с резкими всплесками в январе)
	\item причины энергетических кризисов: изменения налогообложения, рыночные манипуляции, устаревшая инфраструктура, провалы рынка, излишняя зарегулированность, перебои с поставками топлива, резкое изменение климата,  доставка электричества дешевле стоимости производства 
	
\end{itemize}


\end{frame}



\begin{frame}[shrink=5]
\frametitle{Факторы  влияющие на цены на электричество} 
\framesubtitle{Факторы спроса и предложения} 

На равновесие на рынке  электричества влияют: 

\begin{itemize}
	\item  погодные условия (причем при более точном прогнозировании погодных условий можно уменьшить ошибку прогноза цены на электричество)
	\item  уровень  деловой активности  (ежедневной и общего тренда)
	\item  доля ВИЭ и ГП, зависимых от погодных условий
	\item  решения принимаемые  экономическими агентами (при решении оптимизационной задачи)
	\item  цены на ресурсы
	\item  государственная политика, новости
	\item  другие фундаментальные факторы влияющие на баланс спроса и предложения
\end{itemize}

\end{frame}



\begin{frame}[shrink=5]
\frametitle{Факторы  влияющие на цены на электричество} 
\framesubtitle{Несовершенства российского рынка} 

\begin{itemize}
	\item  Высокая степень изношенности основных фондов.
	\item  Перекрестное субсидирование (частичный перенос платежного бремени с населения на промышленность).
	\item  Проблема неплатежей (на конец октября 2017 года на оптовом рынке задолженность составила 65,2 млрд руб., а на розничном — 243 млрд руб).
	\item  Вынужденная генерация  (ТЭЦ  неэффективны на рынке электроэнергии, мощности, работающие в режиме вынужденной генерации, оплачиваются по существенно более высокой цене, чем рыночная).
	\item  Высокие потери тепла.
	\item   Завершение ДПМ и продление ДПМ ВИЭ.
\end{itemize}

\end{frame}


\begin{frame}[shrink=5]
\frametitle{Выбор ценовых факторов } 
\framesubtitle{Несовершенства российского рынка} 

\begin{itemize}
	\item  (***)   изменения в составе ценовых зон
	\item  (***S)   погодные условия 
%	https://www.gismeteo.ru/diary/4517/2014/8/
%	http://rpubs.com/lefkios_paikousis/weatherdata-in-r
%https://rp5.ru/%D0%9F%D0%BE%D0%B3%D0%BE%D0%B4%D0%B0_%D0%B2_%D0%A0%D0%BE%D1%81%D1%81%D0%B8%D0%B8
%https://cran.r-project.org/web/packages/rwunderground/rwunderground.pdf
%https://blog.exploratory.io/riem-package-getting-world-weather-data-in-super-easy-way-78aa94ed45f5
%http://aisori.meteo.ru/climater
	\item  (***TS) уровень деловой активности  
	\item  (**TS) динамика цен на ресурсы и инфляция
	\item  (**TJ) инвестиции в энергетику, завершение ДПМ и продление ДПМ ВИЭ
	\item  (*J) государственная политика, новости
	\item  (*J) аварийность в электросетях и генерации, (ежемесячно или дамми на регионы с высокими рисками нарушения электроснабжения)
%	https://minenergo.gov.ru/node/267
% https://minenergo.gov.ru/node/989
% https://minenergo.gov.ru/node/264
% https://minenergo.gov.ru/node/11200
%	высокая степень изношенности основных фондов
	\item  (*J) проблема неплатежей
	\item  (*) вынужденная генерация
	\item  (*) высокие потери тепла.
	\item  (*) доля ВИЭ и ГП, зависимых от погодных условий
	\item  (*S) коэффициенты сезонности, определенные АТС
\end{itemize}

%http://www.atsenergo.ru/sites/default/files/prognoz/20191128_ishodnye_dannye_i_prognoz_na_2020.pdf

\end{frame}


\begin{frame}[shrink=5]
\frametitle{Выбор ценовых факторов (2) } 


\begin{itemize}
	\item  (***S)   погодные условия; % погодные условия (причем при более точном прогнозировании погодных условий можно уменьшить ошибку прогноза цены на электричество)
	
	%	https://www.gismeteo.ru/diary/4517/2014/8/
	%	http://rpubs.com/lefkios_paikousis/weatherdata-in-r
	%https://rp5.ru/%D0%9F%D0%BE%D0%B3%D0%BE%D0%B4%D0%B0_%D0%B2_%D0%A0%D0%BE%D1%81%D1%81%D0%B8%D0%B8
	%https://cran.r-project.org/web/packages/rwunderground/rwunderground.pdf
	%https://blog.exploratory.io/riem-package-getting-world-weather-data-in-super-easy-way-78aa94ed45f5
	%http://aisori.meteo.ru/climater
	\item  (***TS) уровень деловой активности;   %  (ежедневной и общего тренда)
	
	\item  (**TS) динамика цен на ресурсы и инфляция;
	\item  (**TJ) структура генерирующих мощностей;
	\item  (**TJ) инвестиции в энергетику, завершение ДПМ и продление ДПМ ВИЭ;
	\item  (*J) государственная политика, новости;
	%	\item  (***)   изменения в составе и правилах взаимодействия ценовых зон 
	\item  (*J) аварийность в электросетях и генерации, (ежемесячно или дамми на регионы с высокими рисками нарушения электроснабжения);
	%	https://minenergo.gov.ru/node/267
	% https://minenergo.gov.ru/node/989
	% https://minenergo.gov.ru/node/264
	% https://minenergo.gov.ru/node/11200
	%	высокая степень изношенности основных фондов
	\item  (*J) проблема неплатежей;
	\item  (*) вынужденная генерация;
	\item  (*) высокие потери тепла;
	\item  (*) доля ВИЭ и ГП, зависимых от погодных условий;
	\item  (*S) коэффициенты сезонности, определенные АТС. %  цены имеют три уровня циклических колебаний: ежедневная, недельная, годовая (с резкими всплесками в январе)
	
	
	
	
\end{itemize}

%http://www.atsenergo.ru/sites/default/files/prognoz/20191128_ishodnye_dannye_i_prognoz_na_2020.pdf

\end{frame}



\section{Модели}


\begin{frame}[shrink=5]
\frametitle{Анализ предметной отрасли} 



\begin{table} \small\centering\setlength{\extrarowheight}{0.25em}
	
	\begin{tabular}{   >{\centering\footnotesize}p{7.5em} 
			>{\centering\footnotesize}p{10em}
			>{\centering\footnotesize\arraybackslash}p{18em} }\hline
		
		
		
		Авторы, год & Название работы &  Результат \\\hline 
		
		%		Mehdi Sadeghi, Saeed Shavvalpour (Energy Policy, 2006) & \href{https://www.researchgate.net/publication/222400423_Energy_risk_management_and_value_at_risk_modeling}{Energy risk management and value at risk modeling}  &     \\
		
		
		Judio Lucia, Eduardo Schwartz (Review of Derivatives Research, 2002) & \href{https://link.springer.com/article/10.1023\%2FA\%3A1013846631785}{Electricity Prices and Power Derivatives: Evidence from the Nordic Power Exchange}  &   Эмпирическая оценка детерминистической сезонной компоненты   в одно- и двухфакторной модели  цен на электричество.  \\
		
		
		Álvaro Cartea, Marcelo G. Figueroa (Applied Mathematical Finance, 2005) & \href{https://www.researchgate.net/publication/24071715_Pricing_in_Electricity_Markets_A_Mean_Reverting_Jump_Diffusion_Model_with_Seasonality}{Pricing in Electricity Markets: a mean reverting
			jump diffusion model with seasonality}  &   Применение  модели цен на электричество,  учитывающую тенденцию возвращения к среднему, скачкообразность и сезонность процесса.    \\
		
		
		Maciej Kostrzewski, Jadwiga Kostrzewska (Energy Economics, 2019) & \href{https://www.researchgate.net/publication/331065098_Probabilistic_Electricity_Price_Forecasting_with_Bayesian_Stochastic_Volatility_Models}{Probabilistic Electricity Price Forecasting with Bayesian Stochastic Volatility Models}  &   Прогнозирование спот-цен на электричество с помощью  байесовского подхода  позволяет учесть  неопределенность в распределении коэффициентов параметров, что улучшает прогнозы в сравнении с классическими моделями.  
		
		
		
		%		L. YANG, S. HAMORI (2018) & \href{https://www.worldscientific.com/doi/abs/10.1142/S2010495218500100}{MODELING THE DYNAMICS OF INTERNATIONAL AGRICULTURAL COMMODITY PRICES:A COMPARISON OF GARCH AND SV MODELS}  &    Основываясь на ежемесячных данных, скачкообразные процессы и асимметричный эффект не влияют на цены на сельскохозяйственную продукцию. Оценивая VaR для этих сельскохозяйственных товаров, мы обнаруживаем, что резкий рост цен на сельскохозяйственную продукцию в 2008 году мог быть вызван частой перебалансировкой портфелей. \\
		%		 
		%статья напечатана в говножурнале. Я бы не стал ориентироваться на такое.
		
		
		
		
		%		Mária Bohdalová, Michal Greguš (2015) & \href{https://www.researchgate.net/publication/283243124_ESTIMATING_VALUE-AT-RISK_BASED_ON_NON-NORMAL_DISTRIBUTIONS}{ESTIMATING VALUE-AT-RISK BASED ON NON-NORMAL DISTRIBUTIONS}  &    Моделирование VaR в предположении, что ежедневные изменения цен iid не с нормальным распределением или   автокоррелированны  с через  динамику факторов риска \\
		%  это похоже на пересказ учебника,  к науке и постановке вопроса оно имеет весьма опосредованное отношение.
		
		
		%		Rainer Göb (2011) & \href{https://onlinelibrary.wiley.com/doi/abs/10.1002/qre.1238}{Estimating value at risk and conditional value at risk for count variables}  &    Эмпирические аспекты оценки риска с биномиальным распределением и распределении Пуассона. Особое внимание уделяется интервальной оценке мер риска.  \\\hline 
		
		%			H. Mostafaei et al.  (2013) & 
		%		\href{https://dergipark.org.tr/en/download/article-file/361215}{A Methodology for the Choice of the Best Fitting Continuous-Time Stochastic Models of Crude Oil Price: The Case of Russia }   & Выбор стохастического процесса наилучшим образом, описывающего цену на нефть \\
		%			 A. Murat,  E. Tokat (2009) & \href{https://www.sciencedirect.com/science/article/pii/S0140988308001096?via\%3Dihub}{Forecasting oil price movements with crack spread futures}
		%			   &   Коинтегрированность цен  на нефть и фьючерсов на крэк-спред => прогнозы VECM with crack spread futures лучше, чем у VECM with crude oil futures. 
		
		
		
		
	\end{tabular}
\end{table}


\end{frame}







\begin{frame}[shrink=5]
\frametitle{ Модель Мертона (Merton’s Jump-Diffusion Model) } 

Базовая модель описывающая цену на электричество [Kostrzewski and Kostrzewska (2019)]: 


\begin{itemize}
	\item эмпирическое распределение имеет тяжелые хвосты, что не согласуется со стандартной моделью Блэка-Шоулза
	\item  в модель добавляется отдельная компонента, отвечающая за скачкообразность процесса. 
\end{itemize}

Пусть $S_t$ - цена в момент $t$.

Риск-нейтральный диффузионно-скачкообразный процесс (jump-diffusion process), описывающий изменение цены на электричество:

$$dS_t/S_t=(r−\lambda \bar{k})dt+\sigma dW_t+kdq_t.$$

где $\sigma$ - волатильность диффузионной компоненты, при  $\lambda=0$ получаем модель Блэка-Шоулза.

Скачки порожденны составным процессом Пуассона  $q_t$  с параметром  $\lambda$, где $k$  - размах случайного скачка (логнормально распределенный):    

$$ln(1 +k)\sim N(\gamma,\delta^2)$$

где среднее - $\bar{k} = E(k)=e^{\gamma + \delta^2/2}-1$.








\end{frame}





\begin{frame}[shrink=5]
\frametitle{Модели с детерминистической сезонностью} 

Применение  модели цен на электричество,  учитывающую тенденцию возвращения к среднему, скачкообразность и сезонность процесса.


$$\ln S_t = g(t) +Y_t$$

Детерминистческая компонента - сезонность $g(t)$, стохастическая компонента $Y_t$.


\begin{block}{Lucia and Schwartz (2002): }
	
	$Y_t$ - процесс, возвращающий среднее (OU process)
	
	$$dY_t = -\alpha Y_t dt + \sigma(t) dW_t$$
	
	%Эмпирическая оценка детерминистической сезонной компоненты   в одно- и двухфакторной модели  цен на электричество. 
	
	
	
\end{block}



\begin{block}{Cartea and Figueroa (2005): }
	
	
	$Y_t$ - диффузионно-скачкообразный процесс:
	
	$$dY_t = -\alpha Y_t dt + \sigma(t) dW_t + J dq_t$$
	
	$J$ - величина скачка, $q$ - пуассоновский процесс.
	
	
	
	
\end{block}




\end{frame}

















\begin{frame}[shrink=5]
\frametitle{Kostrzewski and Kostrzewska (2019)} 
\textbf{
Stochastic
	volatility model with a double exponential distribution of jumps, a leverage
	effect and exogenous variables:
}

\begin{itemize}
	\item Цены на электричество зависят от большого числа различных компонент.

\item  Для прогнозирования используется SV модель с экзогенными переменными и дамми-переменными, например, температура, объемы торгов по выходным и понедельникам.

\item  В модели скачки вверх/вниз распределены экспоненциально, с разными параметрами.
\item  С помощью байесовского подхода можно оценить ненаблюдаемые компоненты модели ($q_{ t_{ i}}, \xi_{ t_{i+1}}^D,  \xi_{ t_{i+1}}^U, h_{ t_{ i}} $ ).


\end{itemize}


\end{frame}

\begin{frame}[shrink=5]
\frametitle{Kostrzewski and Kostrzewska (2019)} 

SVDEJX* модель: 
\begin{align*}
y_{t_{i+1}} &= y_{ t_{ i}} + \mu + \psi X_{ t_{i+1} }+ d_{ Sat} D_{ Sat,i+1} + d_{ Sun }D_{ Sun,i+1} +  \\ &+ d_{ Mon }D_{ Mon,i+1} + \sqrt{exp(h_{ t_{ i}} )}\epsilon_{ t_{ i+1} }^{(1)}+ J _{t _{i+1}}\\
h_{ t_{ i+1}} &= h_{ t_{ i}} + \kappa_{ h} (\theta_{ h} − h_{ t_{ i}} ) + \sigma_{ h} (\rho \epsilon_{ t_{ i+1}} + \sqrt{1 - \rho^2} \epsilon_{ t_{ i+1}}^{(2)}) \\
J_{ t_{ i+1}} &= -\xi_{ t_{i+1}}^D \cdot \mathbb{I}(q_{ t_{ i+1}} = -1) + 0 \cdot \mathbb{I} (q_{ t_{ i+1}} = 0) + \xi_{ t_{ i+1}}^U \cdot \mathbb{I} (q_{ t_{ i+1}} = 1)
\end{align*}
$y_{ t_{ i}} = ln(S_{ t_{ i}})$ --  логарифм цены

$h_{ t_{ i}} = y_{ t_{ i+1}} - y_{ t_{ i}}$ --  логарифм цены


$X_{ t_{i+1} }$ -- логарифм почасовой температуры

$D_{ Sat,i+1}, D_{ Sun,i+1}, D_{ Mon,i+1}$  -- учитывают недельную сезонность

$q_{ t_{ i}}$ -- наличие скачка вверх/вниз (значения переменной ненаблюдаемы, но можно оценить вероятность скачка)

$\epsilon_{ t_{ i+1} }^{(1)}, \epsilon_{ t_{ i+1} }^{(2)} \sim N(0,1) \ i.i.d.$, $\xi_{ t_{i+1}}^D \sim Exp(\eta_D) \ i.i.d. $,  
$\xi_{ t_{i+1}}^U \sim Exp(\eta_U) \ i.i.d.$

$\rho < 0$ -- параметр ''рычага'', $\rho > 0$ -- обратный параметр ''рычага'' (если большим значениям логарифма цены  соответствуют большие значения дисперсии)




\end{frame}



\begin{frame}[shrink=5]
\frametitle{Kostrzewski and Kostrzewska (2019)} 

Данные: Спотовые цены JCPL (Jersey Central Power and Light Company), находящейся   в первой  ценовой зоне, определяемой сетевым оператором PJM Interconnection.

08/22, 2010 -  01/14, 2012

\begin{figure}
	\centering
	\includegraphics[width=0.7\linewidth]{screenshot012}
	\caption{Спотовые цены на 4 часа (не-пиковый час) и 16 часов (пиковый час), USD/MWh }
	\label{fig:screenshot006}
\end{figure}



\end{frame}





\begin{frame}[shrink=5]
\frametitle{Kostrzewski and Kostrzewska (2019)} 


\begin{figure}
	\centering
	\includegraphics[width=0.7\linewidth]{screenshot006}
	\caption{Сравнение ширины доверительных интервалов прогноза, полученных по байесовским (B\_Q, B\_HPD) и не байесовским моделям}
	\label{fig:screenshot006}
\end{figure}




\end{frame}






%\begin{frame}[shrink=5]
%\frametitle{Lucia and Schwartz (2002)} 
%
%Данные: 
%
%Nordic Power Exchange's спот-цены
%
%\begin{figure}
%	\centering
%	\includegraphics[width=0.7\linewidth]{screenshot007}
%	\caption{}
%	\label{fig:screenshot007}
%\end{figure}
%
%
%
%\end{frame}
%
%


%\begin{frame}[shrink=5]
%\frametitle{Cartea and Figueroa (2005) } 
%
%Модель: 
%
%Детерминистческая компонента - сезонность:
%
%$$\ln S_t = g(t) +Y_t$$
%
%Стохастическая компонента:
%
%$$dY_t = -\alpha Y_t dt + \sigma(t) dW_t + J dq_t$$
%
%$J$ - величина скачка, $q$ - пуассоновский процесс.
%
%
%\end{frame}

%
%\begin{frame}[shrink=5]
%\frametitle{Lucia and Schwartz (2002)} 
%
%Данные: 
%
%FTSE100; 2/01/90 - 18/06/04. 
%
%\begin{figure}
%	\centering
%	\includegraphics[width=0.7\linewidth]{screenshot008}
%	\caption{}
%	\label{fig:screenshot008}
%\end{figure}
%
%
%
%\end{frame}





















\section{Risks}


\section{Derivatives}


\section{Estimation}



%
%
%\begin{frame}[shrink=5]
%\frametitle{Данные} 
%
%
%
%
%\begin{figure}
%	\centering
%	\includegraphics[width=0.7\linewidth]{screenshot003}
%	\caption{ Ценовые зоны  }
%	\label{fig:screenshot003}
%\end{figure}
%
%\end{frame}
%
%

%
%
%\begin{frame}[shrink=5]
%\frametitle{Данные} 
%
%
%
%
%
%\begin{figure}
%	\centering
%	\includegraphics[width=0.7\linewidth]{screenshot002}
%	\caption{ Почасовые цены в первой и второй ценовых зонах }
%	\label{fig:screenshot002}
%\end{figure}
%
%\end{frame}

%\begin{frame}[shrink=5]
%\frametitle{Специфика российского рынка} 
%
%Примеры:
%
%\begin{itemize}
%	\item 29.03.09 на  фоне снижения спроса отмечено увеличение перетока по контролируемому сечению между ценовыми зонами в сторону Сибири. Отмечено снижение цены в ценовых заявках поставщиков  => принято предложение по наиболее низким ценам 
%	\item   3.06.09 резкое падение индекса равновесных цен по причине снижения спроса на ээ => замыкающими оказались низкие ценовые заявки 
%	\item  13.09.09 снижение потребления электроэнергии вследствие отсутствия заявки на покупку со стороны одного из крупных потребителей => индекс цен снизился
%	
%\end{itemize}
%
%
%\end{frame}
%
%







\end{document}